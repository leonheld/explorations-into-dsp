\documentclass[12pt, a4paper]{report}

\usepackage[T1]{fontenc}
\usepackage[utf8]{inputenc}
\usepackage[portuguese]{babel}
\usepackage{hyphenat}
\usepackage{amsmath}
\usepackage{amssymb}
\usepackage{textcomp}
\usepackage{amsthm}
\newtheorem{theorem}{Theorem}
\newtheorem{definition}{Definition}
\hyphenation{mate-mática recu-perar}

\begin{document}
\begin{titlepage}
    \begin{center}
        \vspace*{1cm}
        \Huge
        \textbf{NOTAS DE PROCESSAMENTO DIGITAL DE SINAIS}
             
        \vspace{3cm}
        EEL7522 \\
        Segundo Semestre Letivo de 2020 
        \vfill
        \Large
        \textbf{Aluno: Leonardo José Held}\\
        \textbf{Professor: Dr. Joceli Mayer}
        \vspace{0.8cm}

        Departamento de Engenharia Elétrica e Eletrônica\\
        Curso de Graduação em Engenharia Eletrônica\\
        Universidade Federal de Santa Catarina\\
        Brasil\\                     
    \end{center}
 \end{titlepage}

 \Large

 \tableofcontents

 \newpage
 \normalsize

 \vspace{1cm}
 Copyleft \textcopyleft  2021 por Leonardo J. Held \\leonardoheld@protonmail.com

\Large
 Permissão para utilizar, copiar, modificar e/ou distribuir este texto e este documento para qualquer propósito com ou sem taxa monetária é permitido.

 \chapter{Módulo 1}

\section{Introdução}
\subsection{Objetivos}

\begin{itemize}
    \item Estudo de sinais e como podem ser utilizados para transmitir, armazenar e processar informação na forma digital.
\end{itemize}

\subsection{Exemplos e aplicações}
\begin{itemize}
    \item Exemplos de sinais: ECG, voz. Sinais geralmente dependem do tempo mas podem ter dependência em outras variáveis.
    \item Possibilidade de vários sensores, gerando informação multidimensional.
    \item Um exemplo de sinal multi(bi-)dimensinal é uma imagem monocromática, onde cada ponto tem duas coordenadas posicionais que servem de input para uma função que indica a luminosidade daquele ponto específico.
    \[\text{Intensidade}(x, y)\]
    \item Outro exemplo é uma foto colorida, que é um sinais bi-dimensional, só que com três canais de cores (como em RGB) sobrepostos.
    \[R(x,y) + G(x, y) + B(x, y)\]
    \item Um vídeo é outro sinal mas com dependência temporal adicionada
    \[R(x, y, t) + G(x, y, t) + B(x, y, t)\]
\end{itemize}

\subsection{Classificação}
\begin{itemize}
    \item Sinais, neste escopo, podem então ser classificados em dimensionalidade e número de canais (canais estes que dependem de variáveis).
    \item Sinais podem ser \textbf{discretos} ou \textbf{contínuos}.
    \item Sinais discretos são definidos apenas para certos pontos na variável dependente.
    \item Um sinal contínuo é definido para todos os pontos na variável dependente.
    
    Sinal Amostrado vs. Sinal Digital:

    \item Sinal Amostrado: discreto no tempo \textbf{contínuo em amplitude}.
    \item Sinal Digital: discreto no tempo \textbf{discreto em amplitude}.
    
    O sinal Digital é um sinal Amostrado e quantizado para apenas seletos possíveis valores de amplitude.

    \item Vale notar que o sinal digital pode ter $n$ quantas de amplitude.
\end{itemize}

\begin{itemize}
    \item 
\end{itemize}

 \chapter{Série de Fourier}
\section{Introdução}

Decomposição de funções numa base exponencial complexa.

\[y[n] = z^{n} \sum^{+\infty}_{k = -\infty}h[k] z^{-k}\]

Ou 

\[y[n] = z^{n}\mathfrak{H}(z)\]

Onde $z^{n}$ é autofunção da transformação linear e $\mathfrak{H}(z) = \sum^{+\infty}_{k = -\infty}h[k] z^{-k}$ é autovalor da transformação linear.

A parte boa é que qualquer sinal pode ser decomposto em sinais exponenciais ponderados

\[x[n] = \sum^{k}_{}a_{k}z^n_k\]

onde a base exponencial é

\[z^n_k = e^{jk \dfrac{2 \pi n}{N}} \]

Lembrando que essa não é a transformação, e sim a série. Logo o sinal representado precisa necessariamente ter um período N.

A diferença do caso contínuo é que no discreto, para essa base específica, existem um conjunto finito de $N$ possíveis, logo se consegue representar um sinal com uma soma finita de exponenciais ponderadas.

\section{Propriedades da Série de Fourier}

Se $x[n] \rightarrow $ períodico, com período fundamental $N$, então

\[x[n] = \sum^{}_{k = <N>}a_{k}e^{jk \dfrac{2 \pi n}{N}}\]

Onde $k$ varia entre $N$ índices quaisquer consecutivos e $a_k$ são os \textbf{coeficientes} da série de Fourier.

\begin{theorem}
    A soma dos valores de uma exponencial complexa em um período é zero a não ser que essa exponencial seja constante.
\end{theorem}

Usando o resultado acima e alguma álgebra (vide caderno, página 31 para dedução) é possível chegar numa fórmula para os coeficientes da série:

\[a_{r} = \dfrac{1}{N} \sum^{}_{n = <N>}x[n]e^{-jr(\dfrac{2 \pi}{N})n}\]

onde o índice $r$ fica dentro da range $<N>$.

Em geral, $a_k = a_{k + n}$, por causa da periodicidade dos coeficientes.

Usando aquele trick de escrever um somatório como multiplicação de matrizes dá pra achar os coeficientes via inversão (ou outros algoritmos de solução de sistema linear matricial).

\section{Transformada de Fourier Discreta a partir da Série de Fourier}

Extensão da série de Fourier para sinal aperiódico.

\begin{equation} 
    X(\Omega) = \sum^{+\infty}_{n = -\infty}x[n]e^{-j\Omega n}
\end{equation}
Possível recuperar $x[n]$ reescrevendo os coeficientes da série em função de uma amostragem com período $\Omega_{0} = \dfrac{2\pi}{N}$ com

\[a_{k} = \dfrac{1}{N} X(k\Omega_{0})\]

Ou

\[\tilde{x}[n] = \dfrac{1}{2\pi} \sum^{}_{k = <N>}X(k\Omega_{0}) \cdot e^{jk\Omega_{0}n}\Omega_{0}\]


Se $N \rightarrow \infty$, então

\[k\Omega_{0} \rightarrow \Omega\]

\[\Omega_{0} \rightarrow d\Omega\]

\[\sum \rightarrow \int\]

De forma que

\begin{equation} 
    x[n] = \frac{1}{2\pi}\int^{}_{2 \pi}X(\Omega)e^{j\Omega n}d\Omega
\end{equation}

onde

\begin{equation} 
    a_{k} = \dfrac{1}{N}X(k\Omega_{0})
\end{equation}
 %\chapter{Conversão A/D, D/A, amostragem de sinais discretos, aumento e redução de taxa, decimação e interpolação}
\section{Conversão de sinais analógicos para digitais}

\begin{itemize}
    \item Padrão de processamento geralmente utilizado: Filtragem $\rightarrow$ C/D $\rightarrow$ Processamento Discreto $\rightarrow$ C/A $\rightarrow$ Filtragem
\end{itemize}

\subsection{A/D}

Modulação em trem de impulsos, a densidade de pulsos depende do período do trem, o que imbuí em uma relação de normalização do indíce pelo período de amostragem.

Se $x_{p}(t) = x(t) p(t)$

\[X_{p}(\omega) = \frac{1}{T}\sum_{k = -\infty}^{+\infty}X(\omega - k\omega_{s})\]

Um conjunto de réplicas do espectro original, que repete com $\omega_{s}$ (que vai impactar a escolha do período).

Se $\omega_{s} > 2 \cdot \omega_{max}$, não há recobrimento entre as réplicas do espectro.

A relação em frequência do sinal amostrado com o sinal contínuo

\[X(\Omega) = X_{p}(\omega) \text{ quando } \omega = \frac{\Omega}{T}\]

\subsection{D/A}

D/A deve fazer o reverso do escalamento realizado pelo A/D de $\Omega = \omega t$

A transformada do pulso amostrador utilizado gera um $sinc$, que possuí distorção em frequência (apesar da fase ser linear).

Se $\tau = t$, é um sample-and-hold :)

\subsection{Amostragem de Sinais Discretos}


\end{document}
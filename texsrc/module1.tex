\chapter{Módulo 1}

\section{Introdução}
\subsection{Objetivos}

\begin{itemize}
    \item Estudo de sinais e como podem ser utilizados para transmitir, armazenar e processar informação na forma digital.
\end{itemize}

\subsection{Exemplos e aplicações}
\begin{itemize}
    \item Exemplos de sinais: ECG, voz. Sinais geralmente dependem do tempo mas podem ter dependência em outras variáveis.
    \item Possibilidade de vários sensores, gerando informação multidimensional.
    \item Um exemplo de sinal multi(bi-)dimensinal é uma imagem monocromática, onde cada ponto tem duas coordenadas posicionais que servem de input para uma função que indica a luminosidade daquele ponto específico.
    \[\text{Intensidade}(x, y)\]
    \item Outro exemplo é uma foto colorida, que é um sinais bi-dimensional, só que com três canais de cores (como em RGB) sobrepostos.
    \[R(x,y) + G(x, y) + B(x, y)\]
    \item Um vídeo é outro sinal mas com dependência temporal adicionada
    \[R(x, y, t) + G(x, y, t) + B(x, y, t)\]
\end{itemize}

\subsection{Classificação}
\begin{itemize}
    \item Sinais, neste escopo, podem então ser classificados em dimensionalidade e número de canais (canais estes que dependem de variáveis).
    \item Sinais podem ser \textbf{discretos} ou \textbf{contínuos}.
    \item Sinais discretos são definidos apenas para certos pontos na variável dependente.
    \item Um sinal contínuo é definido para todos os pontos na variável dependente.
    
    Sinal Amostrado vs. Sinal Digital:

    \item Sinal Amostrado: discreto no tempo \textbf{contínuo em amplitude}.
    \item Sinal Digital: discreto no tempo \textbf{discreto em amplitude}.
    
    O sinal Digital é um sinal Amostrado e quantizado para apenas seletos possíveis valores de amplitude.

    \item Vale notar que o sinal digital pode ter $n$ quantas de amplitude.
\end{itemize}

\begin{itemize}
    \item 
\end{itemize}

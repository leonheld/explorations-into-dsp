\chapter{Módulo 1}

\section{Introdução}
\subsection{Objetivos}

\begin{itemize}
    \item Estudo de sinais e como podem ser utilizados para transmitir, armazenar e processar informação na forma digital.
\end{itemize}

\subsection{Exemplos e aplicações}
\begin{itemize}
    \item Exemplos de sinais: ECG, voz. Sinais geralmente dependem do tempo mas podem ter dependência em outras variáveis.
    \item Possibilidade de vários sensores, gerando informação multidimensional.
    \item Um exemplo de sinal multi(bi-)dimensinal é uma imagem monocromática, onde cada ponto tem duas coordenadas posicionais que servem de input para uma função que indica a luminosidade daquele ponto específico.
    \[\text{Intensidade}(x, y)\]
    \item Outro exemplo é uma foto colorida, que é um sinais bi-dimensional, só que com três canais de cores (como em RGB) sobrepostos.
    \[R(x,y) + G(x, y) + B(x, y)\]
    \item Um vídeo é outro sinal mas com dependência temporal adicionada
    \[R(x, y, t) + G(x, y, t) + B(x, y, t)\]
\end{itemize}

\subsection{Classificação}
\begin{itemize}
    \item Sinais, neste escopo, podem então ser classificados em dimensionalidade e número de canais (canais estes que dependem de variáveis).
    \item Sinais podem ser \textbf{discretos} ou \textbf{contínuos}.
    \item Sinais discretos são definidos apenas para certos pontos na variável dependente.
    \item Um sinal contínuo é definido para todos os pontos na variável dependente.
    
    Sinal Amostrado vs. Sinal Digital:

    \item Sinal Amostrado: discreto no tempo \textbf{contínuo em amplitude}.
    \item Sinal Digital: discreto no tempo \textbf{discreto em amplitude}.
    
    O sinal Digital é um sinal Amostrado e quantizado para apenas seletos possíveis valores de amplitude.

    \item Vale notar que o sinal digital pode ter $n$ quantas de amplitude.
\end{itemize}

\section{Conceitos Básicos de Sinais Discretos}
\subsection{Representação}
\begin{itemize}
    \item Sinais são representados por sequências de amostras (números).
    \[\text{Sinal} = \{x[n]\}, n \in \mathbb{N} \]

    \item $x[0]$, numa representação por sequência deve ser indicada por uma flecha.
    \[x[n] = -1, -2.2, 2_{\uparrow}, 3, 56 \]

    \item Um sinal contínuo pode ser amostrado no tempo, sendo representado por
    
    \[x[n] = x_{a}(t)_{t = nT} = x_{a}(nT) \]

    Onde $T$ denota o período de amostragem. A frequência de amostragem $F_{t}$ é o inverso do período.

    \item Sinal Complexo:
    
    \[\{x[n]\} = \{x_{re}[n]\} + \{x_{im}[n]\} \]
    
    A sequência conjugada é a conjugação de cada termo da sequência.

    \item Sequências podem ser finitas ou infinitas. 
    \item Seja a sequência finita $x[n]$ definida para o intervalo $ N_{1} \leq n \leq N_{2} $, então o comprimento do intervalo será $N_{2} - N_{1} + 1$.

    \item O comprimento da sequência pode ser alterado adicionando zeros.
    
\end{itemize}

\subsection{Norma}
\begin{itemize}
    \item Tamanho do sinal pode ser definido usando representação no espaço $L_{p}$:
    
    \[|x|_{p} = (\sum_{n = - \infty}^{\infty} |x[n]|^{p})^{\dfrac{1}{p}}\]

    onde $p = 2$ dá o RMS do sinal, $p = 1$ é o valor médio absoluto e $p = \infty$ é o valor absoluto de pico da sequência.

\end{itemize}
\subsection{Operações em sequências}
\begin{itemize}
    \item \textbf{Modulação em amplitude}:
    \[y[n] = x[n] \cdot w[n] \]

    Essencialmente o uso da amplitude de um sinal para escalar ou modular a amplitude do outro.

    \item \textbf{Adição}:
    \[y[n] = x[n] + w[n] \]

    \item \textbf{Multiplicação}:
    \[y[n] = A \cdot x[n] \]

    Um ganho no sinal.

    \item \textbf{Deslocamento no tempo}:
    
    \[y[n] = x[n - N] \]

    Para $N > 0$, atraso
    Para $N < 0$, avanço

    
    
\end{itemize}
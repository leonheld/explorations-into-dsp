\chapter{Módulo 1}

\section{Introdução}
\subsection{Objetivos}

\begin{itemize}
    \item Estudo de sinais e como podem ser utilizados para transmitir, armazenar e processar informação na forma digital.
\end{itemize}

\subsection{Exemplos e aplicações}
\begin{itemize}
    \item Exemplos de sinais: ECG, voz. Sinais geralmente dependem do tempo mas podem ter dependência em outras variáveis.
    \item Possibilidade de vários sensores, gerando informação multidimensional.
    \item Um exemplo de sinal multi(bi-)dimensinal é uma imagem monocromática, onde cada ponto tem duas coordenadas posicionais que servem de input para uma função que indica a luminosidade daquele ponto específico.
    \[\text{Intensidade}(x, y)\]
    \item Outro exemplo é uma foto colorida, que é um sinais bi-dimensional, só que com três canais de cores (como em RGB) sobrepostos.
    \[R(x,y) + G(x, y) + B(x, y)\]
    \item Um vídeo é outro sinal mas com dependência temporal adicionada
    \[R(x, y, t) + G(x, y, t) + B(x, y, t)\]
\end{itemize}

\subsection{Classificação em dimensões}
\begin{itemize}
    \item Sinais, neste escopo, podem então ser classificados em dimensionalidade e número de canais (canais estes que dependem de variáveis).
    \item Sinais podem ser \textbf{discretos} ou \textbf{contínuos}.
    \item Sinais discretos são definidos apenas para certos pontos na variável dependente.
    \item Um sinal contínuo é definido para todos os pontos na variável dependente.
    
    Sinal Amostrado vs. Sinal Digital:

    \item Sinal Amostrado: discreto no tempo \textbf{contínuo em amplitude}.
    \item Sinal Digital: discreto no tempo \textbf{discreto em amplitude}.
    
    O sinal Digital é um sinal Amostrado e quantizado para apenas seletos possíveis valores de amplitude.

    \item Vale notar que o sinal digital pode ter $n$ quantas de amplitude.
\end{itemize}

\section{Conceitos Básicos de Sinais Discretos}
\subsection{Representação}
\begin{itemize}
    \item Sinais são representados por sequências de amostras (números).
    \[\text{Sinal} = \{x[n]\}, n \in \mathbb{N} \]

    \item $x[0]$, numa representação por sequência deve ser indicada por uma flecha.
    \[x[n] = -1, -2.2, 2_{\uparrow}, 3, 56 \]

    \item Um sinal contínuo pode ser amostrado no tempo, sendo representado por
    
    \[x[n] = x_{a}(t)_{t = nT} = x_{a}(nT) \]

    Onde $T$ denota o período de amostragem. A frequência de amostragem $F_{t}$ é o inverso do período.

    \item Sinal Complexo:
    
    \[\{x[n]\} = \{x_{re}[n]\} + \{x_{im}[n]\} \]
    
    A sequência conjugada é a conjugação de cada termo da sequência.

    \item Sequências podem ser finitas ou infinitas. 
    \item Seja a sequência finita $x[n]$ definida para o intervalo $ N_{1} \leq n \leq N_{2} $, então o comprimento do intervalo será $N_{2} - N_{1} + 1$.

    \item O comprimento da sequência pode ser alterado adicionando zeros.
    
\end{itemize}

\section{Operações em sequências}
\begin{itemize}
    \item Tamanho do sinal pode ser definido usando representação no espaço $L_{p}$:
    
    \[|x|_{p} = (\sum_{n = - \infty}^{\infty} |x[n]|^{p})^{\dfrac{1}{p}}\]

    onde $p = 2$ dá o RMS do sinal, $p = 1$ é o valor médio absoluto e $p = \infty$ é o valor absoluto de pico da sequência.

\end{itemize}
\subsection{Operações básicas}
\begin{itemize}
    \item \textbf{Modulação em amplitude}:
    \[y[n] = x[n] \cdot w[n] \]

    Essencialmente o uso da amplitude de um sinal para escalar ou modular a amplitude do outro.

    \item \textbf{Adição}:
    \[y[n] = x[n] + w[n] \]

    \item \textbf{Multiplicação}:
    \[y[n] = A \cdot x[n] \]

    Um ganho no sinal.

    \item \textbf{Deslocamento no tempo}:
    
    \[y[n] = x[n - N] \]

    Para $N > 0$, atraso
    Para $N < 0$, avanço

    \item \textbf{Reversão}:
    
    \[y[n] = x[-n]\]

    Como algumas operações requerem um comprimento igual das sequências, pode-se encher a menor com zeros afim de aplicar as operações.

    \item \textbf{Ensemble Average}:
    
    Seja $d_{i}$ um vetor de ruído aditivo aleatório interferindo na i-ésima medida $s$ de algum dado

    \[x_{i} = s + d_{i}\]

    \[ \bar{x} = \frac{1}{K} \sum_{i = 1}^{K}x_{i} = \frac{1}{K} \sum_{i = 1}^{K}s + d_{i} = s + \frac{1}{K} \sum_{i = 1}^{K} d_{i} \]

    O termo $s$ sai do somatório dado que 

    \[\sum_{i = 1}^{K}s = K \cdot s\]

    assumindo que seja um sinal totalmente reprodutível. 
        
\end{itemize}

\subsection{Alteração de taxa de Amostragem}

\begin{itemize}
    \item Adaptar a taxa para interconectar sistemas.
    
    Seja uma sequência $x[n]$ com uma TA $F$. Quer-se adaptar essa taxa gerando um novo sinal $y[n]$ com TA $F'$.

    \item Pode se definir uma Razão de Alteração
    
    \[R = \frac{F'}{F}\]

    Se $R > 1$, tem-se interpolação.

    Se $R > 1$, tem-se decimação.

    \item \textbf{Interpolação}:
    
    Por fator inteiro $L > 1$: Inserção de $L - 1$ amostras com valor zero entre cada ponto da sequência já estabelecido. Um upsampling.

    \item \textbf{Decimação}:
    
    Por fator inteiro $M > 1$: Remoção de $M - 1$ amostras entre cada duas consecutivas da sequência já estabelecida.

    Literalmente matar amostras de maneira periódica.

\end{itemize}

\section{Classificação de sinais}
\subsection{Simetria}
\begin{itemize}
    \item Sequência conjugada-simétrica:
    \[ x[n] = x^{\star}[-n] \]

    e se x[n] for real

    \[ x[n] = x^{\star}[-n] = x[-n] \]

    o que implica em $x[n]$ ser par.

    \item Sequência conjugada-antisimétrica:
    \[ x[n] = -x^{\star}[-n] \]

    e se x[n] for real

    \[ x[n] = -x^{\star}[-n] = -x[-n] \]

    o que implica em $x[n]$ ser ímpar.

    Qualquer soma pode ser escrita pela soma das suas partes conjugada-simétria e antisimétrica.

\end{itemize}

\subsection{Periodicidade}

\begin{itemize}
    \item Se $x[n] = x[n + kN]$ para N inteiro positivo, k inteiro, então é um sinal periódico. O menor N que satisfaça a equação é o período fundamental.
\end{itemize}

\subsection{Energia}

\begin{itemize}
    \item Sinal com energia infinita e potência média finita $\rightarrow$ \textbf{sinal de potência}.
    
    \item Sinal com energia finita e potência média nula $\rightarrow$ \textbf{sinal de energia}.
    
\end{itemize}

\subsection{Delimitação}

\begin{itemize}
    \item Limitação em amplitude se existe uma faixa delimitada de amplitudes.
    \item Uma sequência é absolutamente somável se 
    
    \[\sum_{n = -\infty}{\infty}\]
\end{itemize}

\section{Relações úteis}

\begin{itemize}
    \item \[\delta[n] = u[n] - u[n - 1]\]
    \item \[u[n] = \sum_{m = -\infty}^{n} \]
    \item \[x[n]\delta[n - n_{0} = x[n_{0}]\delta[n - n_{0}]\]
    \item Para se ter um sinal \textbf{discreto} periódico, é necessário que
    
    \[\omega_{0}N = 2\pi r\]

    Nem sempre existem N e r tal que essa relação será comtemplada.

\end{itemize}

\section{Processo de Amostragem}
\begin{itemize}
    \item Sinal de tempo discreto é gerada a partir da amostra de um sinal de tempo contínuo da seguinte forma:
    
    \[x[n] = x(nT)\]
    Quando $t = t_{n} = nT$ implica em
    \[t_{n} = \dfrac{n}{F} = \dfrac{2\pi n}{\Omega}\]

    Onde $F$ denota a frequência de amostragem.

    \item Duas sequências exponenciais podem gerar as mesmas amostras se 
    
    \[\Omega_{1} = \Omega_{0} + 2\pi k\]

    \item Aliasing: diferentes sinais contínuos geram o mesmo sinal discreto devido a baixa taxa de amostragem, gerando perda de informação. No espectro da frequência imbuí numa superposição de espectros.
    \item Para evitar aliasing, frequência de amostragem deve ser maior que a maior frequência do espectro do sinal a ser amostrado.
\end{itemize}
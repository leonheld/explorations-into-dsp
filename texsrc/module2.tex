\chapter{Série de Fourier}
\section{Introdução}

Decomposição de funções numa base exponencial complexa.

\[y[n] = z^{n} \sum^{+\infty}_{k = -\infty}h[k] z^{-k}\]

Ou 

\[y[n] = z^{n}\mathfrak{H}(z)\]

Onde $z^{n}$ é autofunção da transformação linear e $\mathfrak{H}(z) = \sum^{+\infty}_{k = -\infty}h[k] z^{-k}$ é autovalor da transformação linear.

A parte boa é que qualquer sinal pode ser decomposto em sinais exponenciais ponderados

\[x[n] = \sum^{k}_{}a_{k}z^n_k\]

onde a base exponencial é

\[z^n_k = e^{jk \dfrac{2 \pi n}{N}} \]

Lembrando que essa não é a transformação, e sim a série. Logo o sinal representado precisa necessariamente ter um período N.

A diferença do caso contínuo é que no discreto, para essa base específica, existem um conjunto finito de $N$ possíveis, logo se consegue representar um sinal com uma soma finita de exponenciais ponderadas.

\section{Propriedades da Série de Fourier}

Se $x[n] \rightarrow $ períodico, com período fundamental $N$, então

\[x[n] = \sum^{}_{k = <N>}a_{k}e^{jk \dfrac{2 \pi n}{N}}\]

Onde $k$ varia entre $N$ índices quaisquer consecutivos e $a_k$ são os \textbf{coeficientes} da série de Fourier.

\begin{theorem}
    A soma dos valores de uma exponencial complexa em um período é zero a não ser que essa exponencial seja constante.
\end{theorem}

Usando o resultado acima e alguma álgebra (vide caderno, página 31 para dedução) é possível chegar numa fórmula para os coeficientes da série:

\[a_{r} = \dfrac{1}{N} \sum^{}_{n = <N>}x[n]e^{-jr(\dfrac{2 \pi}{N})n}\]

onde o índice $r$ fica dentro da range $<N>$.

Em geral, $a_k = a_{k + n}$, por causa da periodicidade dos coeficientes.

Usando aquele trick de escrever um somatório como multiplicação de matrizes dá pra achar os coeficientes via inversão (ou outros algoritmos de solução de sistema linear matricial).

\section{Transformada de Fourier Discreta a partir da Série de Fourier}

Extensão da série de Fourier para sinal aperiódico.

\begin{equation} 
    X(\Omega) = \sum^{+\infty}_{n = -\infty}x[n]e^{-j\Omega n}
\end{equation}
Possível recuperar $x[n]$ reescrevendo os coeficientes da série em função de uma amostragem com período $\Omega_{0} = \dfrac{2\pi}{N}$ com

\[a_{k} = \dfrac{1}{N} X(k\Omega_{0})\]

Ou

\[\tilde{x}[n] = \dfrac{1}{2\pi} \sum^{}_{k = <N>}X(k\Omega_{0}) \cdot e^{jk\Omega_{0}n}\Omega_{0}\]


Se $N \rightarrow \infty$, então

\[k\Omega_{0} \rightarrow \Omega\]

\[\Omega_{0} \rightarrow d\Omega\]

\[\sum \rightarrow \int\]

De forma que

\begin{equation} 
    x[n] = \frac{1}{2\pi}\int^{}_{2 \pi}X(\Omega)e^{j\Omega n}d\Omega
\end{equation}

onde

\begin{equation} 
    a_{k} = \dfrac{1}{N}X(k\Omega_{0})
\end{equation}
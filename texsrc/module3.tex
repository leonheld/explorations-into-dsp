\chapter{Conversão A/D, D/A, amostragem de sinais discretos, aumento e redução de taxa, decimação e interpolação}
\section{Conversão de sinais analógicos para digitais}

\begin{itemize}
    \item Padrão de processamento geralmente utilizado: Filtragem $\rightarrow$ C/D $\rightarrow$ Processamento Discreto $\rightarrow$ C/A $\rightarrow$ Filtragem
\end{itemize}

\subsection{A/D}

Modulação em trem de impulsos, a densidade de pulsos depende do período do trem, o que imbuí em uma relação de normalização do indíce pelo período de amostragem.

Se $x_{p}(t) = x(t) p(t)$

\[X_{p}(\omega) = \frac{1}{T}\sum_{k = -\infty}^{+\infty}X(\omega - k\omega_{s})\]

Um conjunto de réplicas do espectro original, que repete com $\omega_{s}$ (que vai impactar a escolha do período).

Se $\omega_{s} > 2 \cdot \omega_{max}$, não há recobrimento entre as réplicas do espectro.

A relação em frequência do sinal amostrado com o sinal contínuo

\[X(\Omega) = X_{p}(\omega) \text{ quando } \omega = \frac{\Omega}{T}\]

\subsection{D/A}

D/A deve fazer o reverso do escalamento realizado pelo A/D de $\Omega = \omega t$

A transformada do pulso amostrador utilizado gera um $sinc$, que possuí distorção em frequência (apesar da fase ser linear).

Se $\tau = t$, é um sample-and-hold :)

\subsection{Amostragem de Sinais Discretos}

